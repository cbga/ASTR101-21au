\documentclass[12pt,letterpaper]{article}
\usepackage{fullpage}
\usepackage[top=2cm, bottom=4.5cm, left=2.5cm, right=2.5cm]{geometry}
\usepackage{amsmath,amsthm,amsfonts,amssymb,amscd}
\usepackage{lastpage}
\usepackage{enumerate}
\usepackage{fancyhdr}
\usepackage{mathrsfs}
\usepackage{xcolor}
\usepackage{graphicx}
\usepackage{hyperref}
\usepackage{mathtools}
\usepackage[shortlabels]{enumitem}
\usepackage{esvect}
\usepackage{setspace}
% \usepackage{times}

\hypersetup{%
  colorlinks=true,
  linkcolor=blue,
  linkbordercolor={0 0 1}
}
 
\setlength{\parindent}{0.0in}
\setlength{\parskip}{0.05in}

% Edit these as appropriate
\newcommand\course{ASTR101 Autumn 2021}
\newcommand\hwnumber{\#1}                  % <-- homework number
\newcommand\Name{Name: \textbf{Bingan Chen}} 
\newcommand\id{Student ID: \textbf{180626}}          % <-- NetID of person #1

\pagestyle{fancyplain}
\headheight 35pt
\lhead{\Name \\ \id}
\chead{\textbf{\Large Lab \hwnumber}}
\rhead{\course \\ \today}
\lfoot{}
\cfoot{}
\rfoot{\small\thepage}
\headsep 1.5em


\begin{document}
\section*{Plot 1)}
\begin{center}
    \includegraphics[scale = 0.5]{q1.png}
\end{center}
\begin{enumerate}
    \item The relationship represents that the cubed of the orbital period is proportional to the square of semi-major axis.
    \item Maybe because that even there is a relationship between orbital period and semi-major axis, the constant between cubed period and squared axis is not an exact value. 
    I think this maybe because that the practical situation varies based on some factors in space. A little change in constant could make a huge change in the relationship of the orbital period and semi-major axis.
    \item Because the increase of semi-major axis makes the graph more "discrete". In specific, I think that with even a little change in the semi-major axis, because of its and 
    period's power (2 and 3), could have a enormouse change of period (compared with the theoretical value by formula).
\end{enumerate}
\newpage

\section*{Plot 2)}
\begin{center}
    \includegraphics[scale = 0.5]{q2.png}
\end{center}
The planets of our solar system fit at the part with smaller semi-major 
axis. I found that the eccentricity for planets in exoplanet system 
and semi-major axis is much larger than those in our solar system. 
\newpage

\section*{Plot 3)}
\begin{center}
    \includegraphics[scale = 0.5]{q3.png}
\end{center}

The planets of our solar system would be bellow 0.3 Jupiter mass and those have a shorter semi-major axis than jupiter would 
distribute between 0 to 1.5 AU. However, Uranus, Saturn, and Jupiter would not be displayed 
on the graph. The planets in exoplanet systems seem to possibly be both close to their stars 
and be very massive. In contrast, massive planets in solar system tends to not have 
a short semi-major axis, which I think increases the stability of the system.

\newpage

\section*{Plot 4)}
\begin{center}
    \includegraphics[scale = 0.5]{q4.png}
\end{center}

Because there is the region where our solar system locates, which means that the region around us is where we 
are able to discover more exoplanet using current technologies.

\newpage

\section*{Plot 5)}
\begin{center}
    \includegraphics[scale = 0.5]{q5.png}
\end{center}

As for the planet \textbf{Proxima Centauri b}, its eccentricity is 0.1, which is much larger than that of our Earth (less than 0.02). 
The large eccentricity may results in a very inhabitable climate change. This could make a long-period coldness 
along with another period of extreme hot when the planet is close to its star (Proxima Centauri) that may not 
allow the water's existence in long term.

In terms of the calculated temprature, we can see more clearly that the temprature varies in a range of about 
200 Kelvin (from 117 to 307 K). Unlikely there would be an extreme hot season, but the extreme cold temprature and 
the variation makes it difficult for sustaining lives, though still possible because sometimes the temprature allows 
liquid water to exist.

Now we turns to its mass. The mass of Proxima Centauri b is 0.003691 Jupiter Mass. In comparision, our Earth 
has about 0.003146, which seems pretty similar. This allows a good status of gravitational pull that can hold a 
proper amount of atmosphere, which means the weather on that planet allows primitive bio to develope.

All in all, besides the very eccentric orbital path, it is still possible for life to exist on Proxima Centauri b, 
for its temprature, atmosphere, and gravitational pull.

\newpage

\section*{Plot 6)}

\begin{center}
    \includegraphics[scale = 0.5]{q6.png}
\end{center}

For a chunk of the host stars in this graph, a higher effective temprature corresponds a higher 
mass of the host star. However, at the effective temprature around 4,800 Kelvin, some stars vary   
a lot in mass but with similar temprature. Also, as we oberved at the mass just above 0, some very small 
stars can be extremely hot (up to 10,000 K).

\end{document}