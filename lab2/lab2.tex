\documentclass[12pt,letterpaper]{article}
\usepackage{fullpage}
\usepackage[top=2cm, bottom=4.5cm, left=2.5cm, right=2.5cm]{geometry}
\usepackage{amsmath,amsthm,amsfonts,amssymb,amscd}
\usepackage{lastpage}
\usepackage{enumerate}
\usepackage{fancyhdr}
\usepackage{mathrsfs}
\usepackage{xcolor}
\usepackage{graphicx}
\usepackage{hyperref}
\usepackage{mathtools}
\usepackage[shortlabels]{enumitem}
\usepackage{esvect}
\usepackage{setspace}
% \usepackage{times}

\hypersetup{%
  colorlinks=true,
  linkcolor=blue,
  linkbordercolor={0 0 1}
}
 
\setlength{\parindent}{0.0in}
\setlength{\parskip}{0.05in}

% Edit these as appropriate
\newcommand\course{ASTR101 Autumn 2021}
\newcommand\hwnumber{\#2 Stellar Evolution}                  % <-- homework number
\newcommand\Name{Name: \textbf{Bingan Chen}} 
\newcommand\id{Student ID: \textbf{180626}}          % <-- NetID of person #1

\pagestyle{fancyplain}
\headheight 35pt
\lhead{\Name \\ \id}
\chead{\textbf{\Large Lab \hwnumber}}
\rhead{\course \\ \today}
\lfoot{}
\cfoot{}
\rfoot{\small\thepage}
\headsep 1.5em

\doublespacing

\begin{document}
\begin{center}
    \includegraphics[scale = 0.63]{47Tuc.png}
    \includegraphics[scale = 0.6]{Pleiades.png}
\end{center}
\newpage

\begin{enumerate}
    \item \begin{itemize}
        \item  47 Tucanae \begin{itemize}
            \item MS Turnoff point (B-V): 0.5
            \item Approx age: \(4\times 10^9\) years
        \end{itemize}
        \item Pleiades \begin{itemize}
            \item MS Turnoff point (B-V): \(-0.1\)
            \item Approx age: \(3.2 \times 10^7\) years
        \end{itemize}
    \end{itemize}

    \item It tells me that the stars with same color could have different apparent magnitude 
    with the different distance. In this case, the stars with (B-V) \(\approx 1.0\) at 47 Tuc have a larger apparent 
    magnitude than those stars with same (B-V) value in Pleiades, indicating that stars with (B-V) \(\approx1.0\) in 47 Tuc is 
    more distant than those in Pleiades. Also, because both clusters are much distant 
    from us and we can say that all stars within the same cluster have approximately same 
    distance from us. Thus, we can conclude that 47 Tuc is more distant than Pleiades.

    \item \begin{itemize}
        \item D to 47 Tuc: 
        \[D = 10^{\frac{(m-M+5)}{5}} = 10^{\frac{21-6.6+5}{5}}= 7585.78 \text{ parsecs}\] 
        \item D to Pleiades:
        \[D = 10^{\frac{(m-M+5)}{5}} = 10^{\frac{12.5-6.6+5}{5}}= 151.36 \text{ parsecs}\]
    \end{itemize}

    \item For the distance to 47 Tuc, 
    \[\dfrac{|7586 - 5000|}{5000} \times 100\% = 52\%\]
    my percentage error compared to the distance measured by using parallaxes is about 50\%, which 
    strongly disagrees with the more precise value. \\
    For the distance to Pleiades,
    \[\dfrac{|151 - 180|}{180} \times 100\% = 16\%\]
    my percentage error compared to the distance measured by using parallaxes is about 16\%, which 
    is much closer to the more precise value and more likey agrees with it.

    \item \begin{itemize}
        \item While fusing as red giants, the cores of the them collapse and make in interior 
        heat up again and fuse. But for the surface, it expands and become cooler. Overall, the Red Giant becomes 
        a new equilibrium. This process creates huge radiative and thermal pressure from central region, 
        and makes the red giants have a higher luminosity compared to those young stars in main sequence 
        with similar red color.
        \item As we can see on the graph and the picture of 47 Tuc, it is a globular cluster 
        that has already passed a long lifetime. For O and B stars that are much brighter, the quick 
        fusion rate results in a shorter lifetime compared to those that are dimmer. So for an 
        old cluster, the O and B stars had finished their life cycles and left main sequence.
    \end{itemize}
\end{enumerate}

\end{document}