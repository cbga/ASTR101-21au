\documentclass[12pt,a4paper]{article}
\usepackage{fullpage}
\usepackage[top=2cm, bottom=4.5cm, left=2.5cm, right=2.5cm]{geometry}
\usepackage{graphicx}
\usepackage{ulem}
\usepackage{setspace}
\usepackage{amsmath,amsthm,amsfonts,amssymb,amscd}
\usepackage{mathtools}
\usepackage{siunitx}

\usepackage{array}% in the preamble
\usepackage{xcolor}
\usepackage{pagecolor}
\usepackage{lipsum}  
\usepackage{mdframed}


%\pagecolor{black}
%\color{white}



\doublespacing

\title{Week Quiz Reviw}
\author{Bingan Chen}
\begin{document}
    \maketitle
    \section*{Discussion Materials}
    \subsection*{HR Diagrams}
    \begin{center}
        \includegraphics[scale = 0.4]{hr.png}
    \end{center}
    \subsection*{Star Clusters}
    \subsubsection*{Open Clusters}
    \begin{itemize}
        \item Lots of open space
        \item Gassy and Dusty (light reflects off), bluer, younger, smaller.
    \end{itemize}
    \subsubsection*{Globular Clusters}
    \begin{itemize}
        \item Like a globe ans stars close together
        \item Lots of stars (up to \(10^5\)), little gas/dust (not that blurry), older.
    \end{itemize}
    We assume that all stars relatively same distance from Earth and in same age.
    \subsection*{Main Sequence Trends}
    \begin{itemize}
        \item Mass-Luminosity Relationship: More massive, more luminous.
        \item Mass-Lifetime Relationship: more massive, die quickly.
    \end{itemize}
    \subsubsection*{Turnoff Point}
    Hottest, Brightest star on Cluster's MS is at the turn off point. \textbf{All stars hotter and more massive 
    have already used up H fuel and turned off.} This point moves to less massive stars as the whole 
    cluster ages.
    \begin{center}
        \includegraphics[width = \textwidth]{lifecycle.jpg}
    \end{center}

    \begin{table}[h]
    \centering
    \begin{tabular}{|l|l|}
    \hline
    \emph{LOW Mass} & \emph{HIGH Mass} \\ \hline
    Redder         & Bluer \\ \hline
    Most in Main Sequence  & Less numerous \\ \hline
    Fuse up to \textbf{Carbon}         & Fuse up to \textbf{Iron} \\ \hline
    Die slowly and quietly         &  Die Quickly and Loudly \\ \hline
    Die as \textbf{Planetary Nebula}         & Die as \textbf{Supernova}          \\ \hline
    \end{tabular}
    \end{table}

    \newpage
    \section*{Class Materials}
    \emph{How we observe the stars and find out why many M stars?}\\
    Based on the assumptions mentioned before, we can ignore the effect of distance and compare their 
    \textbf{apparent brightness} as though they were true luminosity. By this, we trust that the \textbf{brighter} 
    stars are more \textbf{luminous}. \\
    All stars formed within 100 million years of each other because of the effects of radiation from newly formed 
    stars on the materials they are forming from. 
    \subsection*{Fusion \& Fision}
    This \textbf{mass energy} can be changed by altering the atom’s mass, via the processes of fission and fusion. 
    \emph{The atoms go into either reactions will not have the same overall mass.} The new atoms produced can 
    have less or more energy, depending on the types of atoms. They can \emph{require energy} because masses are added.\\
    At 1920, suggested that 4 H nuclei can be forced to fuse into 1 He nucleus via \textit{proton-proton chain}. This 
    process generate energy because of the lost mass.\\
    For sun, only small portion of mass is transformed into energy. 4 million tons of mass per sec for 5 billion 
    years, the sun only used up less than 0.1\% of total mass.\\
    Core fusion generate huge \textbf{thermal pressure} creating inward gravity, and the gravity from 
    material outside the core is also strongest at the core. At the surface of the sun, both gravitational 
    and thermal pressure are both the weakest. This called \textbf{gravitational (or hydrostatic) equilibrium}, 
    which sets the Sun's size. \\\\
    All nuclei are positively charged that repel each other. So the nuclei need to be moving fast enough 
    so that the fusion can occur. Only the \textbf{core} of sun have enough energy to overcome the repulsion 
    for the \emph{strong nuclear force}. The higher the temprature is, the faster the fusion rate will be, and more energy 
    produced.\\
    \emph{The region that the production of energy increases, that region expands and cools down. Then 
    the rate slows down and the region shrinks. The shrinking process reheat the region and fasten the 
    fusion again.}\\
    More massive stars need to be supported by the higher inner fusion rate to maintain its gravitational 
    equilibrium (because more force from outer layers need to be balanced).\\
    \emph{This makes the massive stars to burn very quickly even they have more materials.}
    \begin{center}
        \includegraphics[scale = 0.4]{masslife.png}
    \end{center}
    That's why there are fewer stars on the hot end of the MS of those globular clusters. 
    This comes to the \textbf{main sequence turnoff point}. Because the massive stars die quickly.

    \subsection*{Dwarfs, Giants, and Gold}
    The lower limit of main sequence stars - brown dwarfs. They are not massive enough for fusion to happen in core.
    \emph{How they hold themselves stable against gravity?}\\
    It is because of \textbf{degeneracy pressure}, a twist added by the quantum properties of subatomic particles. \\
    With tightly packed electrons, they are forced to move faster into higher energy states. The extra 
    velocity creates a pressure that pushes back against the gravitational forces. It \emph{doesn't need 
    input fuel like fusion and doesn't decay over time}. So objects supported by this force can stay in 
    gravitational equilibrium for unlimited amount of time.\\
    \emph{What happens to stars of different types as this equilibrium evolves?}\\
    The core of star begins to run out and collapse with a decrease in fusion pressure. This makes 
    the interior heat up and the region previously can't fuse, fuses, producing much more heat. This 
    create increase in radiative and thermal pressure from central region, while the outer layers expand and cool 
    to ward a new equilibrium. In to \textbf{Red Giant} phase.\\\\
    The He continues to collapse and reach \(10^8 \si[]{\kelvin}\). Then the \textbf{He fusion} begins, 
    which we called as \textbf{helium flash}. And becomes stable again. These stars are shown in H-R 
    is \textbf{horizontal branch}. They are hotter, dimmer, and smaller as they in Red Giants. \\
    However, this equilibrium stays much shorter than that previous one (thousand times faster). Then 
    the star swells up in size and becomes a \textbf{asymptotic giant branch (AGB)} star, which also called 
    double-shell burning giants.
    \begin{center}
        \includegraphics[scale = 0.6]{dbshell.png}
    \end{center}
    What happends next depends on the star's mass (how much gravitational forces hold the star together).
    \begin{itemize}
        \item less than 8 solar masses: the force are not enough to resist the increase of luminosity, and 
        blows itself apart slowly into \textbf{planetary nebula}. The dust produced during these stages is 
        responsible for most of the carbon in the universe. Not massive enough to heat cores to 
        \(6\times 10^8 \si[]{\kelvin}\), where Carbon can fuse. They do eventually collapse only because of electron 
        degeneracy pressure.
        \item for much massive stars, they have enough gravitational energyto start carbon fusion into Oxygen, 
        while the Carbon used up pretty fast. So the core collapse again when Oxygen itself begins to fuse. \emph{This 
        process repeats to heavier and heavier elements.} As the “shells” of fusion around the core 
        increase in number, thermal and radiative pressure again overbalances 
        the gravity in the outer layers and the surface of the star expands and cools. They become \textbf{supergiant} type stars.\\
        They are unstable, and are categorized as \textbf{luminous blue variables (LBVs)} and \textbf{Wolf-Rayet (WR)}. \\
        Their luminosities push away the outer atmospheres at \(10^{-4}\) solar masses per year (lose 10-90\% of their masses).\\\\
        Their fusion comes to dead end as silicon begins to fuse into iron, because Fe is the lowest mass per 
        nuclear particles of any element, which means \emph{it cannot fuse into another element without creating masses}. Thus, 
        the fusion of Iron \textbf{consumes energy}! 
    \end{itemize}
    With the Fe continues to be made, the density climbs high (\(10^{13}\)) for protons and electrons to form 
    neutrons. This kind of cores no longer supported by electron degeneracy pressure, but by \textbf{neutrons} degeneracy 
    pressure. The collapse takes less than a second from a Earth size to few km. \emph{This heats up a neutron rich core} to 
    temprature up to 100 billion degrees K. The atmosphere above the core falls on to this 
    hot and effectively incompressible surface and recoils, resulting in a shock wave that rips through 
    the outer layers of the star over several hours and sends them blasting outward. \\
    The heat enables \textbf{explosive nucleosynthesis (fusion)} to occur in its atmosphere. The unstable 
    and radioactive elements decay and produce gamma rays heating up the expanding \textbf{supernova remnent} that eventually 
    produces bright debris clouds. \emph{Such steller death sequences are responsible for most of the periodic table.} These 
    are distributed into neaby intersteller space.\\
    In specific, differences are (for high than low):
    \begin{itemize}
        \item less numerous;
        \item shorter lives;
        \item fuse element heavier than Carbon;
        \item die as supernova instead of planetary nebula;
        \item different corpses behind.
    \end{itemize}

\end{document}