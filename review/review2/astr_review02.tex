\documentclass[12pt,a4paper]{article}
\usepackage{fullpage}
\usepackage[top=2cm, bottom=4.5cm, left=2.5cm, right=2.5cm]{geometry}
\usepackage{graphicx}
\usepackage{ulem}
\usepackage{setspace}
\usepackage{amsmath,amsthm,amsfonts,amssymb,amscd}
\usepackage{mathtools}
\usepackage{siunitx}

\usepackage{array}% in the preamble
\usepackage{xcolor}
\usepackage{pagecolor}
\usepackage{lipsum}  
\usepackage{mdframed}


%\pagecolor{black}
%\color{white}



\doublespacing

\title{Week Quiz Reviw}
\author{Bingan Chen}
\begin{document}
    \maketitle
    \section*{Classes Materials}
    \subsection*{White Dwarfs}
    \emph{Degenergy Pressure} keeps Brown Dwarfs to not collapse. It doesn't need fuel. \\
    White dwarfs are very dense and hot on their surfaces. \textbf{Sirius} B is a white dwarf looks 
    very bright under x ray. \emph{White dwarfs is very hot but not bright.} In order to support more mass 
    they need to be dense enough to fit the electrons using degeneracy pressure (\emph{smaller and more massive}). \\
    Once the mass reaches 1.4 \(M_{sun}\), the object cannot be hold by itself only by degenercy pressure. \\
    \subsubsection*{How it be 1.4?}
    \begin{itemize}
        \item White dwarfs don't exceed this mass when they are formed, but if a white dwarf is in a close binary, it can 
        accret masses from its companion.
        \item Hydrogen falls into the white dwarf's surface that is very hot.
        \item Once sufficient pressure builds up, it can fuse explosively, causing the disk to brighten by a factor of 10,000 in 
        just a few days.
        \item Only a small amount of mass are consumed during the nova. 
        \item When its mass becomes above the Chandreasekhar limit;
        \item The temperature is enough to fuse the carbon into heavier elements.
    \end{itemize}
    That calls \textbf{white dwarf supernova} or \textbf{Type Ia supernova}. \emph{This is not the same things as supernova 
    phenomenon for high mass stars.} \\
    The white-dwarf supernovae hit the same peak \emph{luminosity}, true luminosity for one \(=\) all. Then we can not 
    the distance by also using apparent brightness on Earth. The best current way to measure the distance of objects in 
    universe is to measure the distance from a supernova that can be seen 10 billions far.
    \subsection*{Neutron Stars}
    \begin{itemize}
        \item The leftover of Type II supernova including the high mass stars' death.
        \item Their surfaces are \(10^{11}\si[]{\kelvin}\) hot when they were formed.
        \item They were formed in explosio, so they move really fast than other objects.
        \item Small less than \(20\si[]{\kilo\metre}\) across.
        \item Like white dwarfs, they massive they are, the smaller are they in radius.
        \item Supported by neutron degenergy pressure, the limit is higher (about 2.9M\(_{Sun}\)).
        \item Most of them are around 1.3 and 1.5 solar masses.
        \item Dense \(>10^{15}\si[]{\gram\per\centi\metre\cubed}\)
        \item \(G_{gravity}\) can be up yp \(10^{11}\), so escape velocity up to 64\% of the speed of light.
        \item They spin faster when shrinking because of angular momentum. \textit{Its period can be from few seconds 
        to few milliseconds.} 
        \item It creates strong magnatic fields. That creates \textbf{a specific radiation.}
        \item After supernovae of two stars, their remnants finally merge.
    \end{itemize}


    
\end{document}