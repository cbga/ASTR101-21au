\documentclass[12pt,a4paper]{article}
\usepackage{fullpage}
\usepackage[top=2cm, bottom=4.5cm, left=2.5cm, right=2.5cm]{geometry}
\usepackage{graphicx}
\usepackage{ulem}
\usepackage{setspace}
\usepackage{amsmath,amsthm,amsfonts,amssymb,amscd}
\usepackage{mathtools}
\usepackage{siunitx}

\usepackage{array}% in the preamble
\usepackage{xcolor}
\usepackage{pagecolor}
\usepackage{lipsum}  
\usepackage{mdframed}


%\pagecolor{black}
%\color{white}



\doublespacing

\title{Week Quiz Reviw}
\author{Bingan Chen}
\begin{document}
    \maketitle
    \section*{Classes Materials}
    \subsection*{Modern Pic of Galaxy}
    \begin{itemize}
        \item diameter 100,000 l.y.
        \item radius 50,000 l.y.
        \item thickness 1,000 l.y.
        \item number of stars 200 billions
    \end{itemize}

    Half of the visible matter in our galaxy appears to be in stars, and the other half makes up the 
    Intersteller Medium (\textbf{ISM}). It absorbs and scatters visible light, and as a result it masks most of the milly way. 
    We tend to use infrared to abserve galaxy. ISM also radiates energy depending on what stage 
    of the gas-star-gas cycle.

    Supernova emits a lot that can create bubbles around the ISM's hot air. 20\% ~ 50\% are in 
    ISM are bubbles.

    As atomic hydrogen cools further to 10 - 30 \(\si[]{\kelvin}\), it forms \emph{molecular Hydrogen}. 
    The cold clouds' cores collapse into protostars. Stars only form in the place where we have 
    cold gas, which is the center line of galaxy.

    \subsubsection*{Stellar Orbits in the Galaxy}
    Stars in the bulge and halo all orbit the Galactic center, but in randomly distributed directions and inclinations relative 
    to the disk, and with much higher average velocities than stars in the Milky Way's Disk. 
    They do bobble up and down quite a bit though, because of 
    \begin{itemize}
        \item the gravitational pull from nearby objects.
        \item the combined pull of the entire disk.
    \end{itemize}

    The reason why stars orbit around the center of galaxy is because there is a center of 
    mass. This is not because there is a dominent things in the center. 

    There are some dark objects at the center location of the galaxy. We call as Sgr A. Coming within 
    90 AU and reaching a top speed. In order to reach that speed around Sgr A, the mass should be huge as 
    4 millions mass of sun. 

    We are still wondering how the blackhole form.  
    
    Estimate the number of alien species in our galaxy that are capacble of communicating with us. 
    \textbf{Drake's Equation} doesn't really provide the answer but break the question into 
    several parts.
    \begin{enumerate}
        \item \emph{How many habitable planets are there in our galaxy?}\\
        Only those stars at around the suns mass or lower can at least required for intelligent life to exsist. 
        While most stars are lower than the mass of the sun. Proxima Centauri is an M dwarf with a planet in its "habitable zone". 
        (which is common). But moons of Jupiter and Saturn also suggests that the water can exsist on planets on 
        objects out of habitable zone.
        \item \emph{What fraction of those planets ever had life?}\\
        Roughly 1 in 10billion phenomenon that life might exsist. But based on the fossile record cast in the rock shows that 
        simple life arose very quickly after the surface of Earth cooled down. So \textit{simple life} might arise in a much wider range.
        \item \emph{What fraction of those habitable planets ever had intelligent life, capable of communicating.} \\
        Life might arise on many worlds without any species ever evolving into a form capable of space travel or intersteller communication. 

        \item \emph{What fraction of all those that have ever existed, exist \textbf{now}?}\\
        If there have been millions of extraterrestrial space faring civilizationjthroughout Galactiv History, then under all but th emost spocalyptic assumptions 
        many of them stilll exist right now.\\
        We can send self-replacing machines to nearby planets.
    \end{enumerate}

    Solutions are there to Fermi Paradox:

    
\end{document}