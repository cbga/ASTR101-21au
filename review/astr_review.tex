\documentclass[12pt,a4paper]{article}
\usepackage{fullpage}
\usepackage[top=2cm, bottom=4.5cm, left=2.5cm, right=2.5cm]{geometry}
\usepackage{graphicx}
\usepackage{ulem}
\usepackage{setspace}
\usepackage{amsmath,amsthm,amsfonts,amssymb,amscd}
\usepackage{mathtools}
\usepackage{siunitx}



\doublespacing

\title{Week Quiz Reviw}
\author{Bingan Chen}
\begin{document}
    \maketitle
    \section*{Quiz Materials}
    \subsection*{Vocabulary}
    Red is cooler, with longer \(\lambda\) and low frequency / energy; Blue is hotter with shorter \(\lambda\) and high frequency / energy.
    \begin{center}
        \includegraphics[scale = 0.5]{light.png}
    \end{center}
    \textbf{Why need big telescopes?}
    \begin{itemize}
        \item Greater light collection (\emph{photon buckets}) area to see \emph{fainter} objects.
        \item Greater angular resolution to see object in great \emph{detail}.
    \end{itemize}
    \textbf{Ground Based Observations Problems}
    \begin{itemize}
        \item Weather - cloudy, rainy conditions can wreck your schedule.
        \item Distortion - \emph{turbulence} makes the stars appear to twinkle, with low \emph{angular resolution}. 
        \(\Rightarrow\) use \emph{adaptive optics}
        \item Light Pollution - made lights reflected by atmosphere \(\Rightarrow\) the night sky becomes brighter.
        \item Absorption - protect us but not good for science.
        \begin{center}
            \includegraphics[scale = 0.2]{absorption.png}
        \end{center}
    \end{itemize}

    \subsection*{Luminosity \& Apparent Brightness}
    \emph{Luminosity} is the total amount of power radiated by a star in space, as measured at some fixed distance from the star's surface.\\
    \emph{Apprent Brightness} refers to the amount of a star's light which eventually reaches us per unit area hear on Earth.\\
    \textbf{Relationship}: The total radiation from any star is spread out over a greater and greater surface area.
    \[\text{App Brightness} = \dfrac{L}{4\pi d^2}\]
    \subsection*{Distance}
    \textbf{Parallax} is the angular motion of nearby stars due to the Earth's orbit around the sun. (in \emph{arcseconds})\\
    While the distance d is in unit of \emph{parsecs}. (\(1 \text{parsecs} = 3.6 \text{Light Years}\))
    \[d = \dfrac{1}{p}\]
    \begin{center}
        \includegraphics[scale  = 0.7]{para.png}
    \end{center}
    
    \subsection*{Temprature from Spectral Lines}
    We \textit{assume} that all stars have basically the same compostion of H and He. The patterns of \emph{obsorption lines} can reveal the temprature of 
    the stars \emph{much better than color}.
    \begin{center}
        \includegraphics[scale = 0.5]{spec.png}
    \end{center}
    Tempratures in decreasing order: 
    \begin{center}
        \textbf{O B A F G K M}\\
        \includegraphics[scale = 0.3]{color.png}
    \end{center}

    \section*{Lecture Materials}
    \subsection*{Spectrum}
    \emph{Visible Spectrum} is only those wavelength between 400 to 800 nanometers.\\
    All the objects dense enough can \emph{produce} light just like people.\\
    The peak of light for hotter objects lies at higher energy than it does for colder objects.\\
    Also, electrons in any atoms respond in a predictable way to \textit{very particular energies}. \emph{Opaque} hot objects making thermal (continuous) spectra, 
    and \emph{transparent} heated gases making emission (bring line) spectra - interact to produce 
    a third type of spectra, called \emph{absorption spectra}.
    \subsubsection*{Doppler Shift}
    Detected when the pattern of spectral features in an object.
    \subsubsection*{Telescopes}
    Reflector (using lenses) \& Refractor (using mirrors)\\
    Just as I discussed in the first section of this doc.\\
    The development of longer exposure (more than 0.03 seconds) in the late 1800's allowed astronomers to discover 
    the fainter objects.
    \subsection*{Dev of Telescopes}
    The \emph{Adaptive Optics} can de-twinkle the stars \(\Rightarrow\) monitoring the distortions of light from a known and standard source (a laser). Then 
    moving the mirrors millions of times per sec to counter the atmospheric distortions.\\
    Because of the \emph{absorption} of light (only visible radio IR UV reach the ground), the detailed 
    infrared images could be taken with space based telescopes. \emph{This is not because it moves our instruments any closer to the stars.} They are also free from weather and light 
    pollution and effect of turbulent atmosphere. 
    \subsection*{Sun Producing Lights}
    \subsubsection*{History}
    In 1795, William Herschel, discoverer of Uranus, and the person who found there are "nonvisible" light, proposed that sun was cool in its interior, and 
    might support life on its surface. \\
    \subsubsection*{How it works}
    The true story of how sun lights besides simply fusing hydrogen into helium\dots \\
    We need to study more distant stars! What we can measure are
    \begin{itemize}
        \item how bright stars appear to be as seen from Earth;
        \item how much and in what directions they are moving;
        \item their color (\emph{Spectral Energy Distribution});
        \item their absorption / emission lines in their spectra.
    \end{itemize}
    Combined with some physic values
    \begin{center}
        \textit{\textbf{Luminosity, Distance, Mass, Radius, Temperature, Age}}
    \end{center}
    After discussed things before, we can
    \begin{center}
        Measure the luminosity to get distance;\\
        Measure the distance to get luminosity.
    \end{center}
    distance \(\Rightarrow\) \emph{parallax}.\\
    \emph{Mass} can be measured by observing the \emph{binary or multiple} star system (using period and distance from each other) \(\Rightarrow\) to time 
    the details of \emph{ellipse} to know how long it takes for one star to fully move in front of other and how long it takes to fully cross (both size are revealed).\\
    We comes to the important equation
    \[L = 4\pi R^{2} \sigma T^4\]
    We mostly estimate the radius by assuming the luminosity is based on its temprature alone and use the law of thermal radiation. 
    So the \emph{temprature} is needed because of the effect on \(T^4\).\\
    \subsubsection*{Temprature of Stars}
    Roughly on color, precisely through details of \emph{steller spectra}.\\
    Spetral type is not determined by stars compostion. All of them contains \emph{75\% H}, \emph{23\% He} and \emph{1-2\%} heavier elements. 
    (1925 by Cecilia Payne-Gaposchkin)\\
    She showed that the \emph{temprature} of star's atmosphere dictates the types of \emph{ions or molecules} that can exist there. This is with a precision of 10 deg of K, much better than color.


\end{document}